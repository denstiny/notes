% Created 2021-03-25 四 19:39
% Intended LaTeX compiler: pdflatex
\documentclass[11pt]{article}
\usepackage[utf8]{inputenc}
\usepackage[T1]{fontenc}
\usepackage{graphicx}
\usepackage{grffile}
\usepackage{longtable}
\usepackage{wrapfig}
\usepackage{rotating}
\usepackage[normalem]{ulem}
\usepackage{amsmath}
\usepackage{textcomp}
\usepackage{amssymb}
\usepackage{capt-of}
\usepackage{hyperref}
\author{aero}
\date{\today}
\title{}
\hypersetup{
 pdfauthor={aero},
 pdftitle={},
 pdfkeywords={},
 pdfsubject={},
 pdfcreator={Emacs 27.1 (Org mode 9.3)}, 
 pdflang={English}}
\begin{document}

\tableofcontents

\section{java 笔记}
\label{sec:org8b7b832}
\subsection{基本数据类型}
\label{sec:orgbd96cf9}

\begin{center}
\begin{tabular}{cr}
\hline
类型 & 类型\\
int & 整形\\
byte & 短整形\\
shot & 整形\\
long & 长整形\\
\hline
float & 浮点数\\
double & 双精准度\\
\hline
char & 字符型\\
\hline
boolde & \\
\hline
\end{tabular}
\end{center}

\begin{enumerate}
\item Code
\label{sec:org5f5b745}

\begin{verbatim}
	/**
    * @author Jackrabbit
    * @version 1.0
    * @date 2021/3/22 14:18
   */

public class Study_Variable {
    /*
    1、常量:是不可变化量,数字
    2、变量:可以改变的量,函数里的 x,y。生活中的年龄等
    3、标识符:是用来标识一些特定意义的符号。变量名称,常量名称,类名等
    4、标识符的命名规则:
	1):只能由 英文字母,数字,_(下划线),$ 组成
	2):数字不可以开头
	备注:大驼峰命名法--每个单词的首字母要大写,比如Study_Variable,HelloWorld
		    一般用在类名,接口等
	      小驼峰命名法--首字母小写,后续所有的单词首字母大写,xiaoMing
		       变量,方法名
     */
    //main 函数,是Java程序的入口
    /*
    这个是我学习到的第一个Java的代码,是一个函数
    这个函数是程序运行的入口
    这个函数很重要,需要默写下来
     */
    /**
     *
     * @param args
     */
    public static void main(String[] args){
//        整数:
//        在Java中 整数是默认为 int 如果需要以long存储,则需要在数字的后面加 L
	byte b1 = -128;
	short s1 = 129;
	int i1 = -2100000000;
	long l1 = -2300000000L;

	//小数,在java中,小数默认是以double形式存在,如果要存入float,则需要在数字后面加f
	float f1 = 1.2f;
	double d1 = 1.2;

	//字符型 char,用单引号,只能存放一个字符(字母)
	char c1 = 'q';
	char c2 ='中';
	//转义符:将java中一些特殊的字符进行转义(转换意义)
	char c3 = '\'';//  \' 代表真实的字符  '
	System.out.println(c3);
	//布尔型
	boolean bool1= true;
	boolean bool2 = false;


	//数据类型转换
	/*
	整数之间数据的转换
	byte --> short --->int ---> long
	long --> int--->short --->byte
	 */

	byte x = 127;
	short x1 = x;
	int x2 = x1;
	long x3 = x2;
	System.out.println("x-->"+x);
	System.out.println("x1-->"+x1);
	System.out.println("x2-->"+x2);
	System.out.println("x3-->"+x3);
	long y = 2200000000L;
	int y1 = (int) y;
	short y2 =(short) y1;
	byte y3 =(byte) y2;
	System.out.println("y--->"+y);
	System.out.println("y1--->"+y1);
	System.out.println("y2--->"+y2);
	System.out.println("y3--->"+y3);
	/*
	float --->double
	double ---->float
	 */
	float f_1 = 12.34f;
	double d_1 = f_1;
	System.out.println("f_1--->"+f_1);
	System.out.println("d_1--->"+d_1);
	double d_2 = 123.457789;
	float f_2 = (float) d_2;
	System.out.println("f_2--->"+f_2);
	System.out.println("d_2--->"+d_2);


	int i_1 = 1234;
	double d_3 = i_1;

	System.out.println(d_3);

	double d_4 = 12345.2345;
	int i_4 =(int) d_4;
	System.out.println(i_4);

	char c_5 = 'Q';
	System.out.println(c_5+1);
	int i_5 = c_5;
	System.out.println(i_5);
	/*
	String 是 引用数据类型----类(String 类)
	用双引号标识
	 */
	String str = "这个是一个字符串";
	System.out.println(str);
	String str1 = "1+";
	String str2 = "2";
	String str3 = "=3";
	System.out.println(str1+1+'A'+false+123.4);



    }



}
\end{verbatim}
\end{enumerate}





\subsection{运算符}
\label{sec:orgf8d8497}


\begin{center}
\begin{tabular}{llll}
tab & tab & tab & \\
\hline
单目运算符 & 名称 & 示例 & 分类\\
\hline
<c> & <c> & <c> & 双目\\
+ & 加法 & A=A+B & \\
- & 减法 & A=A-B & \\
\% & 取于 & A\%b & \\
\hline
+ & 正号 & +A & \\
- & 负号 & -A & 单目\\
++ & 自增 & ++A or A-- & \\
-- & 自减 & --A or A-- & \\
\hline
\end{tabular}
\end{center}









\begin{verbatim}
#include<stdio.h>
int main() {
 printf("hello world");
 return 0;
}
#+END_SRC :

#+RESULTS:
: hello world





#+BEGIN_SRC java
public class Test {
    public static void main(String[] args) {
	long startTime1 = System.nanoTime();
	for (int i = 0; i < 1000000; i++) {

	}
	System.out.println("i++ time(ms):" + Long.toString((System.nanoTime() - startTime1) / 1000));
	System.out.println("i++ time(s):" + Long.toString((System.nanoTime() - startTime1) / 1000 / 1000));

	long startTime2 = System.nanoTime();
	for (int i = 0; i < 1000000; ++i) {

	}
	System.out.println("++i time(ms):" + Long.toString((System.nanoTime() - startTime2) / 1000));
	System.out.println("++i time(s):" + Long.toString((System.nanoTime() - startTime2) / 1000 / 1000));
    }


    }


\end{verbatim}




\begin{center}
\includegraphics[width=.9\linewidth]{/home/aero/Imager/1.jpg}
\end{center}
\end{document}
